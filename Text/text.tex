\documentclass{article}
\usepackage[utf8]{inputenc}
\usepackage{graphicx}
\usepackage{geometry}
\geometry{a4paper, total={16cm, 24cm}, top=2cm}
\usepackage{amsmath}
\usepackage{blindtext}
\graphicspath{ {img/} }
\usepackage{listings}
\usepackage{color}
\usepackage{siunitx}
\usepackage{hyperref}
\hypersetup{
    colorlinks=true,
    linkcolor=blue,
    filecolor=magenta,      
    urlcolor=cyan,
    }
\definecolor{dkgreen}{rgb}{0,0.6,0}
\definecolor{gray}{rgb}{0.5,0.5,0.5}
\definecolor{mauve}{rgb}{0.58,0,0.82}

\lstset{ %frame=tb,
  language=C,
  aboveskip=3mm,
  belowskip=3mm,
  showstringspaces=false,
  columns=flexible,
  numbers=left,
  basicstyle={\small\ttfamily},
  numberstyle=\tiny\color{gray},
  keywordstyle=\color{blue},
  commentstyle=\color{dkgreen},
  stringstyle=\color{mauve},
  breaklines=true,
  breakatwhitespace=true,
  tabsize=3  
}


\title{Week 6 Programming Assignment}
\author{Steffen Petersen | au722120}
\date{October 10th 2022}

\begin{document}
%\tableofcontents

\maketitle
\vspace{5pt}
\noindent Here is the link for my repository, in which you will find all the edited code files and such.\\
\url{https://github.com/Aarhus-University-ECE/assignment-6-SirQuacc}
\section{}
\includegraphics[width=\linewidth, keepaspectratio=true]{task1}
\vspace{2pt}\\
The program would print: x=7, y=18, z=2
\pagebreak

\section{}
\includegraphics[width=\linewidth, keepaspectratio=true]{task2}
\vspace{2pt}
Below is the function itself, it can also be found in max.c
\begin{lstlisting}
  int max(int* numbers, int size)
  {
      assert(size > 0);
      int highest = numbers[0]; // Initialize the "highest" variable as the first element in the given array to start.
      for(int i = 1; i < size; i++){ // Run through the array from 1 -> n-1
        if(numbers[i] > highest) // Test whether the current slot in the array has a higher value than the current saved highest value.
          highest = numbers[i]; // If above is true, highest is reassigned as the value of the current spot in the array.
      }
      return highest;
  }
\end{lstlisting}
\pagebreak
\section{}
\includegraphics[width=\linewidth, keepaspectratio=true]{task31} 
\includegraphics[width=\linewidth, keepaspectratio=true]{task32} 
Here is my diagram showing the list at the two /*show list here*/ places in the code.\\
\begin{center} \includegraphics[scale=0.6, keepaspectratio=true]{showlists}
\end{center}
\vspace{10cm}
\includegraphics[width=\linewidth, keepaspectratio=true]{task33b} 
The code is seen below and can be found in list.c
\begin{lstlisting}
  int size(node *l){
    assert(l!=NULL); // Same precond, input needs to exist
    int count = 0; // Start counter for size at 0
    while(l->next!=NULL){ 
      l = l->next; //Progress through the list
      count++; //while the current node's "next" pointer is defined, count 1 up.
    }
    return count;
  }
\end{lstlisting}
\vspace{1cm}
\includegraphics[width=\linewidth, keepaspectratio=true]{task33c}
No the code does not fulful the precondition, because the while condition
only checks whether the current node p exists, and also it doesn't ever progress through the list.\\
\includegraphics[width=\linewidth, keepaspectratio=true]{task33d}
The function's while condition would need to check $\si{p->next}$ instead of just p, and it would have to reassign $\si{p = p->next}$ before finishing the while loop.\\
The corrected function is below, and in list.c
\begin{lstlisting}
  void printout(node *l) {
      node *p = l->next;
      while (p->next!=NULL){ //Check if we are at the final node
        printf("%d\n, ",p->data);
        p = p->next; //Continue to next node.
      }
  }
\end{lstlisting}
\vspace{3cm}
\includegraphics[width=\linewidth, keepaspectratio=true]{task33e}
Below is my code for this function, it can of course also be found in list.c
\begin{lstlisting}
  int largest(node *l){
    /*Excercise 3e) Add your code below.*/
    assert(l!=NULL); //assert l is defined
    if (l->next == NULL) return -1; //Return -1 if nothing is in the input list
    node *p = l->next; //Create local pointer p, that points to the same as inputs next
    int i = 0; //Counter variable
    int *dataElements = malloc(sizeof(int) * size(l)); //Allocate memory for data elements in an int array of the same size as the input's length.
    for(int i = 0; i < size(l); i++){
      dataElements[i] = p->data; //Input current data into data array
      if(p->next != NULL) p = p->next; //If not at final node, continue
    }
    int highest = max(dataElements, size(l)); // Call our max function to get the highest value in the int array of data.
    return highest; 
     /*pre: head points to the first, empty element. The last element's next is NULL. size(l>0)
      post: returns the largest value of the list*/
  }
\end{lstlisting}


\vspace{2pt}


\end{document}